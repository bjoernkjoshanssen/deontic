\newcommand{\fivecweak}{
	If $Y\in\ob(X)$
	and $Z\in\ob(X)$
	and $X\cap Y\cap Z\ne\emptyset$,
	then $Y\cap Z\in\ob(X)$.
}
\newcommand{\fiveeweak}{
	If $Y\subseteq X$
	and $Z\in\ob(X)$
	and $Y\cap Z\ne\emptyset$,
	and $\overline Z\not\in\ob(Y)$,
	then $Z\in\ob(Y)$.
}

\newcommand{\fivef}{
	If $X\in\ob(Y)$ and $X\in\ob(Z)$ then $X\in\ob(Y\cup Z)$.
}
\newcommand{\fiveg}{
	If $Y\in\ob(X)$ and $Z\in\ob(Y)$ and $X\cap Y\cap Z\ne\emptyset$, then $Y\cap Z\in\ob(X)$. (A form of transitivity for obligations.)
}


\newcommand{\fivee}{
	If $Y\subseteq X$ and $Z\in\ob(X)$ and $Y\cap Z\ne\emptyset$, then $Z\in\ob(Y)$.
}
\newcommand{\fived}{
	If $Y\subseteq X$ and $Y\in\ob(X)$ and $X\subseteq Z$, then $(Z\setminus X)\cup Y\in\ob(Z)$.
}
\newcommand{\fivec}{
	If $Y\in\ob(X)$ and $Z\in\ob(X)$ then $Y\cap Z\in\ob(X)$.
}
\newcommand{\fiveb}{
	If $Y\cap X=Z\cap X$ then $Y\in\ob(X)$ iff $Z\in\ob(X)$.
}
\newcommand{\fivea}{
	$\emptyset\not\in\ob(X)$.
}
Carmo and Jones in 2022 \cite{MR4500520} proposed certain axioms 5(a)--(g) for a relation $\ob$ that holds between sets of possible worlds $X$ and $Y$ if $X$ is obligatory in the context $Y$. It was the latest iteration in a sequence of systems \cite{CJ96,CJ02,MR3063042}.

We will exhibit a paradox therein.
Our paradox will be a weak form of conditional deontic explosion: given that something is somewhat desirable (passing a course with a grade of C, say) and given that the most desirable outcome (the grade of A) is unavailable, the somewhat desirable outcome becomes obligatory.

We then show that despite this paradox, the systems of Carmo and Jones have interesting mathematical content. For the strongest system we provide a full classification of its models; for weaker versions we characterize the least models (under inclusion) satisfying the axioms and basic contrary-to-duty assumptions.

\chapter{The paradox}
	Let $W$ be a finite set of possible worlds of a given model, and let $\mathscr P$ denote the power set operation.

 %    \textsc{Carmo and Jones}\footnote{
	% 	Conditions 5(a)--(d) and 5(e) are first introduced on pages 331 and 341, respectively, of~\cite{CJ96}, where ob is called pi.
	% 	5(a)--(d) are also given in~\cite{CJ02}, page 291, with 5(e) on page 319.
	% 	The conditions 5(a)(b)(d)(e) and a condition (c$^*$) are given in~\cite{CJ13}, page 590.
	% } gave several
    
   Combining the systems from several papers, the full list of Carmo and Jones'
    conditions
	on a function $ob : \mathscr P(W) \to \mathscr P(\mathscr P(W))$ is as follows. 
	\begin{itemize}
		 \item[5(a)] Axiom 5(a): \fivea (Obligations cannot be impossible to fulfil.)
		\item[5(b)] Axiom 5(b): \fiveb
		(Whether $Z$ is obligatory in context $X$ depends only on $Z$ through $Z\cap X$.)
         \item[5(c)] Axiom 5(c$^-$) (2002, page 319; finite version of 5(c$^*$), 2013, which is called 5(c) in 2022): \fivecweak (Closure under intersection.)
         \item[5(c)] Axiom 5(c) (1997 and 2002, page 287): \fivec (Closure under intersection, even for disjoint sets.)
		 \item[5(d)] Axiom 5(d): \fived (If $Y$ is obligatory in context $X$ then a form of the material conditional ``$X \to Y$'' holds in a wider context.)
		\item[5(e)] Axiom 5(e): \fivee
        (If $Z$ is obligatory in the context $X$ then it remains obligatory in any subcontext in which it is still possible.)
	\item[5(f)] Axiom 5(f): \fivef
    (If $X$ is obligatory in both of the contexts $Y$ and $Z$, then it is obligatory in their union.)
	\item[5(g)] Axiom 5(g): \fiveg
    \end{itemize}

% In this file you should put the actual content of the blueprint.
% It will be used both by the web and the print version.
% It should *not* include the \begin{document}
%
% If you want to split the blueprint content into several files then
% the current file can be a simple sequence of \input. Otherwise It
% can start with a \section or \chapter for instance.

Carmo and Jones gave several axioms on their function $\mathrm{ob}$.


\begin{definition}\label{5(a)}
	\lean{A5}
	\leanok
	Axiom 5(a) for a function $\ob: \Pow(U) \to \Pow(\Pow(U))$ says that $\emptyset\notin\ob(X)$ for all $X\in\Pow(U)$.
\end{definition}



\begin{definition}\label{5(b)}
	\lean{B5}
	\leanok
	Axiom 5(b) for a function $\ob: \Pow(U) \to \Pow(\Pow(U))$ says that
	\[
	∀ X\, Y\, Z, Z ∩ X = Y ∩ X \to (Z ∈ \ob X ↔ Y ∈ \ob X)
	\]

\end{definition}

\begin{definition}\label{5(c) weak}
	\lean{C5}
	\leanok
	The weak, but potentially infinite, version of Axiom 5(c) for a function $\ob: \Pow(U) \to \Pow(\Pow(U))$ says that
	\[
    \forall X, \forall \beta \subseteq \ob X, \beta \ne \emptyset \rightarrow \bigcap \beta \cap X \ne \emptyset
	\rightarrow \bigcap \beta \in \ob X.	
	\]

\end{definition}

\begin{definition}\label{5(c) strong}
	\lean{C5Strong}
	\leanok
	The strong version of Axiom 5(c) for a function $\ob: \Pow(U) \to \Pow(\Pow(U))$ says that
	\fivec
\end{definition}
 
 
\begin{definition}\label{5(d)}
	\lean{D5}
	\leanok
	Axiom 5(d) for a function $\ob: \Pow(U) \to \Pow(\Pow(U))$ says that
	\[
	  ∀ (X, Y, Z\subseteq U), Y ⊆ X \to Y ∈ ob X \to X ⊆ Z \to Z \ X ∪ Y ∈ ob Z
	\]

\end{definition}

\begin{lemma}\label{insert_single_ob_pair}
	\lean{insert_single_ob_pair}
 	\leanok
	\uses{5(d)}
	If $A\in\ob(X)$ whenever $A\subseteq X$ and $a_1\ne a_2$ then $A\cup\{a_1\}\in\ob(A\cup\{a_1,a_2\})$.
\end{lemma}

% \begin{definition}\label{5(bd)}
% 	\lean{CJ5bd}
% 	\leanok
% 	Axiom 5(bd) for a function $\ob: \Pow(U) \to \Pow(\Pow(U))$ says that
% 	$\forall (X Y Z \in\Pow U), Y \in \ob X \land X \subseteq Z \to Z \setminus X \cup Y \in \ob Z$.
% \end{definition}

\begin{definition}\label{5(e)}
	\lean{E5}
	\leanok
	Axiom 5(e) for a function $\ob: \Pow(U) \to \Pow(\Pow(U))$ says that
\fivee
\end{definition}



\paragraph{The argument.} Axioms 5(b)(e)(f) will now be used to derive a paradox.

Suppose that James is taking an exam on which the possible grades, in decreasing order of quality, are A, B, C, D and F.

If seems reasonable to assume that given that James' grade is A or B, it ought to be A.

Moreover, given that the grade is C or D, it ought to be C.

Then by axiom 5(b), given that the grade is C or D, it ought to be A or C.
Moreover, also by 5(b), given that the grade is A or B, it ought to be A or C.
In other words, James ought to make sure the proposition ``the grade is A or C'' is true (of course, ideally by getting an A).

It also follows that the grade ought to be A or C given that it is A, B, C, or D, this time using axiom 5(f) applied to the two previous statements. (If this is getting hard to follow, the reader may rest assured that a formalization is available, see \ref{sa}.)

So far this is not entirely unintuitive. But now we use 5(e), and conclude that since A or C was obligatory in the context A, B, C, or D it must remain obligatory in a more restrictive context in which it is still possible, namely B, C, or D.

Finally, using 5(b) again, we conclude that given that the grade is B, C, or D, it ought to be C. While there is such a concept as ``gentleman's C'', surely this is a paradox.\footnote{Traditionally a \emph{Gentleman's C} is given to someone who deserves D or F but for extraneous reasons is deemed worthy of a C. Here this is reversed, as a C is deemed obligatory even when a B is available.
}



\paragraph{A literary perspective.} 
Let us consider this paradox above in another context, beside students and their grades.

Suppose James has the following options.
\begin{enumerate}[A]
\item Marriage to Alice, his favorite.
\item Marriage to Alice's sister Beatrice, or one of several other women he also quite likes.
\item Remaining a lonely bachelor.
\item Marriage to Deirdre, whom he despises.
\end{enumerate}

Then our conclusion is like that of the Bee Gees in 1977:
\begin{quote}
If I can't have you, I don't want nobody.    
\end{quote}

% This sentiment is, it seems to us, inherently surprising, perhaps even hyperbolic. Still, it retains some plausibility, and is relatively entertaining, compared to alternative statements such as
% \begin{itemize}
%     \item \emph{If I can't have you, I'll marry someone I despise} (too unjustified)
%     \item \emph{If I can't have you, I'll settle for Beatrice} (not dramatic enough)
% \end{itemize}
% It is a limited deontic explosion, in contrast with the full deontic explosion of Romeo's statement in \emph{Romeo and Juliet}:
% \begin{quote}
% There is no world without Verona walls but purgatory, torture, hell itself.
% \end{quote}


\paragraph{From despondency to mathematics.}
Hage \cite{Hage2000-HAGDNE} thinks the task Carmo and Jones and others have set themselves is impossible.

From Hage's book review, in the context of discussing a man who (i) should visit his neighbor, (ii) if he does visit, should call to say that he is coming, and (iii) is not in fact coming:

\begin{quote}
	[...] Actually, A does not assist his neighbours and
	therefore should not call them. The two intuitions presuppose a different role for
	deontic logic, namely, reasoning about what is ideally the case, and reasoning about
	what ought to be done in the real world. These two roles are hard to reconcile in
	one logic, unless the logic combines the two kinds of reasoning in different parts.
	The paper by Carmo and Jones in the present volume provides such a combination
	logic.
	Nevertheless, many have attempted to do what is, in my opinion, \textbf{undoable}\footnote{[emphasis ours]} and
	this has lead to many modifications of the Standard System of Deontic Logic
	(Hilpinen, 1971, p. 13f.; Chellas, 1980, p. 190f.). These new systems have lead
	to new paradoxes that deal with CTD obligations.
\end{quote}


Given Hage's sentiment
and the paradox we have presented, one might temporarily become despondent. If deontic systems always have flaws, why pursue them? However, in the course of uncovering this paradox I also discovered interesting \emph{mathematical structure} in the axioms 5(a)--5(g).

\begin{definition}\label{least}
    Given a set $W$ and a family $\mathcal F$ of functions $f :\mathscr P(W)\to\mathscr P(\mathscr P(W))$, we say that $f_0\in\mathcal F$ is the \emph{least} element of $\mathcal F$ if for all $f\in\mathcal F$ and all $X\subseteq W$, $f_0(X)\subseteq f(X)$.

    The \emph{least model} of a set of axioms $\mathscr A$ concerning a variable function $\mathrm{ob} :\mathscr P(W)\to\mathscr P(\mathscr P(W))$ is the least element of the collection $\mathcal F$ of all functions $\mathrm{ob} :\mathscr P(W)\to\mathscr P(\mathscr P(W))$ satisfying all axioms in $\mathscr A$.

    ``Axiom'' here is used to mean simply a condition on $\mathrm{ob}$, although we may note that all the axioms 5(a)--(g) may be formulated in first-order set theory.
\end{definition}

The least model of 5(b) given certain ``oughts'' has a very natural characterization, as does the least model of 5(b), 5(d), and 5(f).

To be specific let us write $\mathrm{Ought}(A\mid B)$ to mean that for each $X\subseteq B$, if $A\cap X\ne \emptyset$ then $A\in\ob X$. This is the semantic condition used by Carmo and Jones for the conditional obligation operator $O(A\mid B)$. A pair of oughts $(\mathrm{Ought}(A\mid W)$, $\mathrm{Ought}(B\mid W \setminus A))$ forms a basic contrary-to-duty obligation of $B$ given that our duty $A$ has failed to be observed.

\begin{definition}\label{canon₂}
\lean{canon₂}
\leanok
canon₂ A B X = if X ∩ B = ∅ then ∅, else: if X ∩ A = ∅ then $\{T \mid X \cap B \subseteq T\}$ else $\{T \mid X \cap A \subseteq T\}$.
\end{definition}

\begin{definition}\label{canon₂_II}
\lean{canon₂_II}
\leanok

canon₂_II A B X = if X ∩ B = ∅ then ∅, else: if X ∩ A = ∅ then $\{Y \mid X \cap B = X \cap Y\}$ else $\{Y \mid X \cap A = X \cap Y\}$.
\end{definition}
In other words, if $Y \in\ob (X)$ under a canon₂_II model then generically $X\cap Y$ consists exactly of the most desirable worlds in $X$.

\begin{theorem}\label{characterize_canon₂_II}
	\lean{characterize_canon₂_II}
	\leanok
The least model $\ob$ (under inclusion) of $\mathrm{Ought}(A\mid W)$, $\mathrm{Ought}(B\mid W \setminus A)$, and axiom 5(b) is canon₂_II A B.
\end{theorem}

Even though the model arises from assuming 5(b) only, it also satisfies axioms 5(a), 5(c), 5(e) and 5(g). This indicates a certain robustness of our definitions.


\begin{theorem}\label{characterize_canon₂}
	\lean{characterize_canon₂}
	\leanok
The least model of axioms 5(b), 5(d), 5(f) and the two ``oughts'' in \ref{t1} is canon₂ A B.
\end{theorem}
In other words, $X\cap Y$ contains at least all the most desirable worlds in $X$.
Since axiom 5(f) follows from 5(a)(b)(c)(d), as shown by Carmo and Jones, the model can alternatively be characterized as the least one satisfying the latter four axioms and the two specified Oughts.

The two families of models in \ref{characterize_canon₂_II} and \ref{characterize_canon₂} represent two alternative approaches to contrary-to-duty obligations as discussed in \cite{MR3607634} (called I and II there). They do not exhaust the interesting models by any means: for instance, we may have conflicting obligations. This may lead us to prefer 5(c) to 5(c)*, in particular. For a concrete example, suppose that James has received acceptances on separate marriage proposals to both Alice and Beatrice, but cannot marry them both.

\paragraph{Classification.}
In mathematics, classification theorems are fairly common. For example, the finite abelian groups have a straightforward characterization and the finite simple groups a complicated one. Vector spaces over a fixed field $\mathbb F$ of finite dimension are characterized by their dimension $d$, hence the single parameter $d$ determines the space up to isomorphism.

In deontic logic, where axioms and rules are added based on moral intuitions, we should perhaps not expect structures that are mathematically natural enough to be classifiable.

However, for the full theory of Carmo and Jones's axioms 5(a)--5(e) with the strong version 5(c)* in which we do not impose nondisjointness, we can characterize its models completely. The only nontrivial ones basically say that there is just one bad world and the only obligation is to avoid it:
\begin{theorem}\label{models_ofCJ_1997_equiv}
	\lean{models_ofCJ_1997_equiv}
	\leanok
    Let $W$ be a finite set of possible worlds and let $\ob : \mathscr P (W)\to \mathscr P(\mathscr P(W))$. The the following are equivalent:
    \begin{enumerate}
        \item $\ob$ satisfies the full system suggested in \cite{CJ96}: 5(a), (b), (c)*, (d) and (e).
        \item One of the following three holds:
        \begin{enumerate}
            \item $\ob(X)=\emptyset$ for all $X$.
            \item $\ob(X)=\{Y \mid \emptyset \ne X \subseteq Y\}$ for all $X$.
            \item There is a distinguished possible world $a$ (the ``forbidden'' world) such that for all $X$,
            $\ob(X)=\{Y \mid X \cap Y \ne \emptyset \text{ and } X\setminus \{a\}\subseteq Y\}$.
        \end{enumerate}
    \end{enumerate}
\end{theorem}
Details on the proof of Theorem \ref{models_ofCJ_1997_equiv} can be found in the next section.


\chapter{Technical details of the characterization of CJ97}

\begin{definition}\label{BDE}
	\lean{BDE}
	\leanok
	\uses{5(b),5(d),5(e)}
	The theory BDE consists of the axioms 5(b)(d)(e).
\end{definition}	

\begin{definition}\label{ADE}
	\lean{ADE}
	\leanok
	\uses{5(a),5(d),5(e)}
	The theory ADE consists of the axioms 5(a)(d)(e).
\end{definition}	



\begin{definition}\label{CX}
	\lean{CX}
	\leanok
	Conditional explosion for ob is the statement that
	∀ (A B C : Finset U), A ∈ ob C → B ∩ Aᶜ ∩ C ≠ ∅ → B ∈ ob (Aᶜ ∩ C).
\end{definition}
\begin{theorem}\label{conditional_explosion}
	\lean{conditional_explosion}
	\leanok
	\uses{ADE,BDE}
    If ob satisfies axioms 5(a)(b)(d)(e) then ob satisfies conditional explosion.
\end{theorem}

% \begin{definition}\label{ABCDE}
% 	\lean{ABCDE}
% 	\leanok
% 	\uses{ADE, BDE, 5(c) strong}
% 	The theory ABCDE consists of all the axioms 5(a)(b)(d)(e) and 5(c) strong.
% \end{definition}

\begin{lemma}\label{single_ob_pair}
	\lean{single_ob_pair}
	\leanok
	\uses{insert_single_ob_pair,5(c) weak,5(e)}
	If $a_1\ne a_2$, $a_1\notin A$, $a_2\notin A$, $\{a_1,a_2\}\in\ob(\{a_1,a_2\})$,
	and $A\in\ob(X)$ whenever $A\subseteq X$, then
	    $\{a_1\} \in \ob \{a_1, a_2\}$ and $\{a_2\} \in \ob \{a_1, a_2\}$.
\end{lemma}

\begin{lemma}\label{semiglobal_holds}
	\lean{semiglobal_holds}
	% \leanok
	\uses{single_ob_pair, 5(c) strong}
	If $a_1\ne a_2$, $a_1\notin A$, $a_2\notin A$, $\{a_1,a_2\}\in\ob(\{a_1,a_2\})$,
	then it cannot be that $A\in\ob(X)$ whenever $A\subseteq X$.
\end{lemma}

\begin{lemma}
	\label{global_holds_specific}
	\lean{global_holds_specific}
	\leanok
	\uses{semiglobal_holds,5(a),5(b),conditional_explosion}
	If  $a_1 \notin A$ and $a_2 \notin A$ and $a_1 \ne a_2$ then it cannot be that $A\in\ob(X)$ whenever $A\subseteq X$.
\end{lemma}

% define 5(bde) as a theory?

% \begin{lemma}\label{obSelf_of_bad_not_mem}
% 	\lean{obSelf_of_bad_not_mem}
% 	\leanok
% 	\uses{BDE}
% 	If $a$ is $\ob$-bad, $a\notin Y \ne \emptyset$, then $Y\in\ob(Y)$.
% \end{lemma}

\begin{definition}\label{bad}
\lean{bad}
\leanok
A world $a$ is bad if  ∃ (X : Finset (Fin n)), a ∈ X ∧ $X \setminus \{a\} \in \ob X$.
\end{definition}
\begin{definition}\label{quasibad}
\lean{quasibad}
\leanok
The world $a$ is quasibad if ∃ (X : Finset (Fin n)) (Y : Finset (Fin n)), a ∈ X $\setminus$ Y ∧ Y ∈ ob X.
\end{definition}

Thus, a world $a$ is bad if in some context there is an obligation to simply avoid $a$.
For example, if there is an obligation ``do not go to war'' then the world representing ``going to war with Syria''
is \emph{quasibad}, but it is not \emph{bad} unless there is also the specific obligation ``do not go to war with Syria''. (In ``reasonable'' this distinction would perhaps not need to be made, but here we are in the process of proving that a certain system 5(abcde) is not reasonable.)

\begin{lemma}\label{obSelfSdiff_of_bad}
    \lean{obSelfSdiff_of_bad}
	\leanok
	\uses{BDE}
	If $a$ is bad and $Y \setminus \{a\} \ne \emptyset$ then
    $Y \setminus \{a\} \in \ob Y$.
	\end{lemma}

Lemma \ref{obSelfSdiff_of_bad} says that badness of the world does not depend on context.

\begin{lemma}\label{obSelf_of_obSelf}
	\lean{obSelf_of_obSelf}
	\leanok
    \uses{BDE}
	If  $X ∈ \ob X$ and $Y ≠ ∅$ then $Y ∈ ob Y $.
\end{lemma}

Lemma \ref{obSelf_of_obSelf} says that if any context is obligatory relative to itself, then they all are.

\begin{lemma}\label{obSelf_of_obSelfSdiff}
	\lean{obSelf_of_obSelfSdiff}
	\leanok
	\uses{obSelf_of_obSelf}
	
	    % (a5 : A5 ob) (b5 : B5 ob) (d5 : D5 ob) (e5 : E5 ob)
	    % {X : Finset (Fin k)} {a : Fin k}
	If $\emptyset \ne X \setminus \{a\} ∈ ob X$ then $X ∈ \ob X$.
\end{lemma}

Lemma \ref{obSelf_of_obSelfSdiff} says that if there is a bad world then the corresponding context is self-obligatory.

% \begin{lemma}\label{ob_singleton_of_obSelf}
% 	\lean{ob_singleton_of_obSelf}
% 	\leanok
% 	\uses{5(a),5(b),5(c) weak}
% 	    (a5 : A5 ob) (b5 : B5 ob) (c5 : C5 ob) {a : Fin k}
% 	    If $\{a\} ∈ \ob \{a\}$ then $\ob \{a\} = \{Y \mid a ∈ Y\}$.
% \end{lemma}

\begin{lemma}\label{obSelf_of_bad.single}
	\lean{obSelf_of_bad.single}
	\leanok
	\uses{BDE,obSelf_univ}
	    % (b5 : B5 ob) (d5 : D5 ob) (e5 : E5 ob)
		% Assume that $∀ a, \mathrm{univ} \setminus \{a\} ∈ \ob \mathrm{univ} \to \mathrm{univ} ∈ \ob \mathrm{univ}$.
Assume axioms 5(abde).
If $a$ is $\ob$-bad then $\{a\} ∈ \ob \{a\}$.
\end{lemma}

Lemma \ref{obSelf_of_bad.single} is a technicality: even if $a$ is bad, it is still self-obligatory.

\begin{lemma}\label{obSelf_univ}
	\lean{obSelf_univ}
	\leanok
	\uses{ADE}
	    % (a5 : A5 ob) (d5 : D5 ob) (e5 : E5 ob) (a : Fin k)
	If $\mathrm{univ} \setminus \{a\} ∈ \ob \mathrm{univ}$ then $\mathrm{univ} ∈ \ob \mathrm{univ}$.
\end{lemma}

Lemma \ref{obSelf_univ} is another technicality: if $a$ is bad in the global context then the global context is self-obligatory.

\begin{lemma}\label{local_of_global}
	\lean{local_of_global}
	\leanok
	\uses{ADE}
    % (a5 : A5 ob) (d5 : D5 ob) (e5 : E5 ob)
	Suppose that for all contexts $A$, if $A$ is obligatory in all larger contexts $X\supseteq A$, then $A$ is a cosubsingleton, i.e., missing at most one element (from the global context).
	Then for all $B$ and $C$, if $B\subseteq C$ is obligatory relative to $C$ then $C\setminus B$ is a cosubsingleton.

    % If for all $A$, $\mathrm{inter_ifsub} \ob A → \mathrm{cosubsingleton} A$,
    % then covering ob.
\end{lemma}

Lemma \ref{local_of_global} is a ``global-to-local'' principle allowing us to conclude a fact about an arbitrary context $C$ from
a fact about the global context.
% \begin{theorem}\label{bd5}
% 	\lean{bd5}
% 	\leanok
% 	\uses{CJ5bd, CJ5b, CJ5d}
% 	\uses{D5}
% 	If $\ob$ satisfies 5(b) and 5(d) then it satisfies 5(bd).
% \end{theorem}

The antecedent of Lemma \ref{local_of_global} is provided by Lemma \ref{global_holds}
and hence the consequent is provided by Lemma \ref{local_holds}.
\begin{lemma}\label{global_holds}
	\lean{global_holds}
	\leanok
	\uses{global_holds_specific}
	For all contexts $A$, if $A$ is obligatory in all larger contexts $X\supseteq A$, then $A$ is a cosubsingleton, i.e., missing at most one element (from the global context).
\end{lemma}

\begin{lemma}\label{local_holds}
	\lean{local_holds}
	\leanok
	\uses{global_holds, local_of_global}
   
	For all $B$ and $C$, if $B\subseteq C$ is obligatory relative to $C$ then $C\setminus B$ is a cosubsingleton.
\end{lemma}

\begin{definition}\label{stayAlive}
	\lean{stayAlive}
	\leanok
The model \emph{stayAlive} is defined by
stayAlive e X = $\{Y  | X \cap Y \ne \emptyset \wedge X \setminus \{e\} \subseteq  X \cap Y\}$.
\end{definition}
	
	
\begin{definition}\label{alive}
	\lean{alive}
	\leanok
	
The model \emph{alive} is defined by

	alive n X = $\{Y \mid X \ne \emptyset \wedge Y \supseteq X\}$.
\end{definition}

\begin{definition}\label{noObligations}
	\lean{noObligations}
	\leanok

The model \emph{noObligations} is defined by

	noObligations X = ∅.

\end{definition}

We think of \emph{alive} as a computer game like ``Snake'' where the objective is to stay alive,
with the surprising twist that it is not possible to die. In contrast, in the model \emph{noObligations} there are no obligations at all, and in the model \emph{stayAlive} the objective is standard: stay alive.

Thus, someone playing Snake under the noObligations model can relax completely, whereas someone playing under the \emph{alive} model may worry
that perhaps there is a way to die that they just have not seen yet. In fact, \emph{alive} is a reduct of \emph{stayAlive} where we remove the one bad world.

We prove several technical lemmas, culminating in Theorem \ref{models_ofCJ_1997}:
\begin{lemma}\label{alive_of_no_quasibad}
	\lean{alive_of_no_quasibad}
	\leanok
	\uses{local_holds,obSelf_of_obSelf}
	If ob ≠ noObligations and ∀ a, ¬ quasibad ob a, then
	    ob = alive k.
\end{lemma}
	
\begin{lemma}\label{unique_bad}
	\lean{unique_bad}
	\leanok
	\uses{local_holds}
    % (a5 : A5 ob) (b5 : B5 ob) (c5 : C5Strong ob) (d5 : D5 ob) (e5 : E5 ob)
    If bad ob a and bad ob b then $a = b$.
\end{lemma}

\begin{lemma}\label{bad_cosubsingleton_of_ob}
	\lean{bad_cosubsingleton_of_ob}
	\leanok
	\uses{unique_bad}
	If bad ob a and $X ∩ Y ∈ ob X$ then
	    $X ∩ Y = X$ or $X ∩ Y = X \setminus \{a\}$.
\end{lemma}

% \begin{lemma}\label{obSelf_of_bad_mem}
% 	\lean{obSelf_of_bad_mem}
% 	\leanok
% 	\uses{obSelfSdiff_of_bad,obSelf_of_obSelfSdiff,5(a)}
% If semibad ob b and $¬X = \{b\}$ and $b ∈ X$ then
%     $X ∈ ob X$.
% \end{lemma}

\begin{lemma}\label{obSelf_of_bad.nonsingle}
	\lean{obSelf_of_bad.nonsingle}
	\leanok
	\uses{obSelf_of_obSelfSdiff,obSelfSdiff_of_bad}
If bad ob a and X ≠ {a} and X ≠ ∅ then X ∈ ob X.
\end{lemma}


\begin{lemma}\label{obSelf_of_bad}
	\lean{obSelf_of_bad}
	\leanok
	\uses{obSelf_of_bad.single,obSelf_of_bad.nonsingle}
    If bad ob a and $X \setminus \{a\}\ne \emptyset$ then $X ∈ \ob X$.
\end{lemma}


%
%
% \begin{lemma}\label{ob_bad}
% 	\lean{ob_bad}
% 	\leanok
% 	\uses{obSelfSingleton_of_bad,obSelf_univ,ob_singleton_of_obSelf}
% If bad ob a then $\ob \{a\} = \{Y \mid a ∈ Y\}$.
%
% \end{lemma}

% \begin{lemma}\label{bad_of_semibad}
% 	\lean{bad_of_semibad}
% 	\leanok
% 	\uses{5(a), 5(e)}
%
% If semibad ob a then bad ob a.
% \end{lemma}

\begin{lemma}\label{sub_stayAlive_of_bad}
	\lean{sub_stayAlive_of_bad}
	\leanok
    \uses{bad_cosubsingleton_of_ob}
	If bad ob a then ∀ Y, ob Y ⊆ stayAlive a Y.
\end{lemma}

Note that Lemma \ref{stayAlive_sub_of_bad} does not require any form of C5.
\begin{lemma}\label{stayAlive_sub_of_bad}
	\lean{stayAlive_sub_of_bad}
	\leanok
	\uses{obSelf_of_bad,obSelfSdiff_of_bad}
    If bad ob a then ∀ Y, stayAlive a Y ⊆ ob Y.
\end{lemma}

\begin{lemma}\label{stayAlive_of_bad}
	\lean{stayAlive_of_bad}
	\leanok
	\uses{stayAlive_sub_of_bad, sub_stayAlive_of_bad}
	If bad ob a then ob = stayAlive a.
\end{lemma}


\begin{theorem}\label{models_ofCJ_1997}
	\lean{models_ofCJ_1997}
	\leanok
	\uses{stayAlive_of_bad,alive_of_no_quasibad}
	Every model of axioms 5(abcde) is either stayAlive $a$ for some bad world $a$, alive, or noObligations.
\end{theorem}

% \begin{definition}\label{CJ5f}
% 	\lean{CJ5f}
% 	\leanok
% 	Axiom 5(f) says
% 	\[
% 		\forall \beta \subseteq \Pow(U), \beta \ne \emptyset \rightarrow
% 		\forall X, \text{ if} \forall Z \in \beta, X \in \ob Z \text{ then } X \in \ob \bigcup \beta.
% 	\]
% \end{definition}
% While 5(f) allows for infinite sets, the finite version can be stated more simply.

% \begin{theorem}[Lemma II.2.2 of Carmo and Jones]\label{II22}
% 	\lean{II_2_2}
% 	\leanok
% 	\uses{CJ5a, CJ5b, CJ5cstar,CJ5d}
% 	If $\ob$ satisfies axioms 5(a,b,c*,d) then $\ob$ satisfies axiom 5(f).
% \end{theorem}

% \begin{definition}\label{32models}
% 	\lean{ob₂}
% 	\leanok
% The 32 models $\ob_2(b)$, where $b$ is a Boolean vector of length 5, are defined by $B\in\ob_2(A)$ iff
% the following are true:
% \begin{itemize}
% 	\item $A \cap B \ne \emptyset$,
% 	\item $(B = \{0\} \to b 0)$,
% 	\item $(B = \{1\} \to b 1)$, and
% 	\item if $B = \{0,1\}$ then
% 		\begin{itemize}
% 			\item $(A = \{0\} \to b 3)$,
% 			\item $(A = \{1\} \to b 4)$, and
% 			\item $(A = \{0,1\} \to b 2)$.
% 		\end{itemize}
% \end{itemize}
% \end{definition}

% Using this ob₂ we prove that
% \begin{theorem}\label{not_imply_5df}
% 	\lean{do_not_imply_5d_or_5f}
% 	\uses{32models}
% 	\leanok
% There is a model where $5(a)(b)(c)(e)(g)$ all hold but $5(d)$ and $5(f)$
% do not hold.
% \end{theorem}

% We also prove
% \begin{theorem}\label{not_imply_5e}
% \lean{do_not_imply_5e}
% \leanok
% ∃ ob, A5 ob ∧ B5 ob ∧ C5 ob ∧ D5 ob ∧ ¬ E5 ob ∧ F5 ob ∧ G5 ob.
% \end{theorem}

% \section{Canonical models of Carmo and Jones' systems}
%
% We show that the two approaches sketched in [Kjos-Hanssen 2017] are both consistent with [Carmo Jones 2022]
% \cite{MR4500520}.
%
% Preferably, we let F(X) = X ∩ A for a fixed set A.
%
% However, to incorporate contrary-to-duty obligations we introduce a predicate B,
% where A worlds, A ⊆ B, are the best and B \setminus A worlds the second best.
%
% Thus, if X ∩ A = ∅ but X ∩ B ≠ ∅, we let F(X) = X ∩ B.
%
%
%
% \begin{table}[h]
%     \centering
%     \begin{tabular}{lcccc}
%         \toprule
%         Axiom \textbackslash\ Model & \texttt{canon} & \texttt{canon\_II} & \texttt{canon₂} & \texttt{canon₂\_II} \\
%         \midrule
%         A & ✓ & ✓ & ✓ & ✓ \\
%         B & ✓ & ✓ & ✓ & ✓ \\
%         C & ✓ & ✓ & ✓ & ✓ \\
%         D & \textit{thus} ✓ & × & ✓ & \textit{thus} × \\
%         E & × & ✓ & \textit{thus} × & ✓ \\
%         F & ✓ & ✓ & ✓ & ×! \\
%         G & ✓ & ✓ & ×! & ✓ \\
%         \bottomrule
%     \end{tabular}
%     \caption{Results about which axioms hold in which model.}
%     \label{tab:example}
% \end{table}

% \begin{definition}\label{canon}
% 	\lean{canon}
% 	\leanok
% Given n : ℕ and A : Finset (Fin n)),
% T $\in$ canon(A,S) iff S ∩ A $\ne$ ∅ and S ∩ A ⊆ T.
% \end{definition}
\section{Acknowledgments}\label{sa}

All mathematical claims above are verified in the proof assistant \textsc{Lean}, see \cite{deontic}.

The main argument for our paradox was discovered by carefully analyzing some output from a script in the computer mathematics system \textsc{Maple} by Maplesoft \cite[Appendix]{K96}.

This work was partially supported by a grant from the Simons Foundation (\#704836 to Bj{\o}rn Kjos-Hanssen).

\bibliographystyle{plain}
\bibliography{deontic}
