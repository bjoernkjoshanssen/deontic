% In this file you should put the actual content of the blueprint.
% It will be used both by the web and the print version.
% It should *not* include the \begin{document}
%
% If you want to split the blueprint content into several files then
% the current file can be a simple sequence of \input. Otherwise It
% can start with a \section or \chapter for instance.
This Lean blueprint project \emph{Deontic} formalizes the paper \emph{Natural models of Carmo and Jones axioms for contrary-to-duty obligations}.

Carmo and Jones gave several axioms on their function $\mathrm{ob}$.


\begin{definition}\label{CJ5a}
	\lean{CJ5a}
	\leanok
	Axiom 5(a) for a function $\ob: \Pow(U) \to \Pow(\Pow(U))$ says that $\emptyset\notin\ob(X)$ for all $X\in\Pow(U)$.
\end{definition}



\begin{definition}\label{CJ5b}
	\lean{CJ5b}
	\leanok
	Axiom 5(b) for a function $\ob: \Pow(U) \to \Pow(\Pow(U))$ says that
	\[
	∀ X\, Y\, Z, Z ∩ X = Y ∩ X → (Z ∈ \ob X ↔ Y ∈ \ob X)
	\]

\end{definition}

\begin{definition}\label{CJ5cstar}
	\lean{CJ5c_star}
	\leanok
	Axiom 5(c*) for a function $\ob: \Pow(U) \to \Pow(\Pow(U))$ says that
	\[
    \forall X, \forall \beta \subseteq \ob X, \beta \ne \emptyset \rightarrow \bigcap \beta \cap X \ne \emptyset
	\rightarrow \bigcap \beta \in \ob X.	
	\]

\end{definition}
 
 
\begin{definition}\label{CJ5d}
	\lean{CJ5d}
	\leanok
	Axiom 5(d) for a function $\ob: \Pow(U) \to \Pow(\Pow(U))$ says that
	\[
	  ∀ (X Y Z : Set U), Y ⊆ X → Y ∈ ob X → X ⊆ Z → Z \ X ∪ Y ∈ ob Z
	\]

\end{definition}
 
 
\begin{definition}\label{CJ5bd}
	\lean{CJ5bd}
	\leanok
	Axiom 5(bd) for a function $\ob: \Pow(U) \to \Pow(\Pow(U))$ says that
	$\forall (X Y Z \in\Pow U), Y \in \ob X \land X \subseteq Z \to Z \setminus X \cup Y \in \ob Z$.
\end{definition}


\begin{theorem}\label{bd5}
	\lean{bd5}
	\leanok
	\uses{CJ5bd, CJ5b, CJ5d}
	If $\ob$ satisfies 5(b) and 5(d) then it satisfies 5(bd).
\end{theorem}

\begin{definition}\label{CJ5f}
	\lean{CJ5f}
	\leanok
	Axiom 5(f) says
	\[
		\forall \beta \subseteq \Pow(U), \beta \ne \emptyset \rightarrow
		\forall X, \text{ if} \forall Z \in \beta, X \in \ob Z \text{ then } X \in \ob \bigcup \beta.
	\]
\end{definition}
While 5(f) allows for infinite sets, the finite version can be stated more simply.

\begin{theorem}[Lemma II.2.2 of Carmo and Jones]\label{II22}
	\lean{II_2_2}
	\leanok
	\uses{CJ5a, CJ5b, CJ5cstar,CJ5d}
	If $\ob$ satisfies axioms 5(a,b,c*,d) then $\ob$ satisfies axiom 5(f).
\end{theorem}