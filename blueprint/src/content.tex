\newcommand{\fivecweak}{
	If $Y\in\ob(X)$
	and $Z\in\ob(X)$
	and $X\cap Y\cap Z\ne\emptyset$,
	then $Y\cap Z\in\ob(X)$.
}
\newcommand{\fiveeweak}{
	If $Y\subseteq X$
	and $Z\in\ob(X)$
	and $Y\cap Z\ne\emptyset$,
	and $\overline Z\not\in\ob(Y)$,
	then $Z\in\ob(Y)$.
}

\newcommand{\fivef}{
	If $X\in\ob(Y)$ and $X\in\ob(Z)$ then $X\in\ob(Y\cup Z)$.
}
\newcommand{\fiveg}{
	If $Y\in\ob(X)$ and $Z\in\ob(Y)$ and $X\cap Y\cap Z\ne\emptyset$, then $Y\cap Z\in\ob(X)$. (A form of transitivity for obligations.)
}


\newcommand{\fivee}{
	If $Y\subseteq X$ and $Z\in\ob(X)$ and $Y\cap Z\ne\emptyset$, then $Z\in\ob(Y)$.
}
\newcommand{\fived}{
	If $Y\subseteq X$ and $Y\in\ob(X)$ and $X\subseteq Z$, then $(Z\setminus X)\cup Y\in\ob(Z)$.
}
\newcommand{\fivec}{
	If $Y\in\ob(X)$ and $Z\in\ob(X)$ then $Y\cap Z\in\ob(X)$.
}
\newcommand{\fiveb}{
	If $Y\cap X=Z\cap X$ then $Y\in\ob(X)$ iff $Z\in\ob(X)$.
}
\newcommand{\fivea}{
	$\emptyset\not\in\ob(X)$.
}
Carmo and Jones in 2022 \cite{MR4500520} proposed certains axioms 5(a)--(g) for a relation $\ob$ that holds between sets of possible worlds $X$ and $Y$ if $X$ is obligatory in the context $Y$. It was the latest iteration in a sequence of systems \cite{CJ96,CJ02,MR3063042}.

We will exhibit a paradox therein.
Our paradox will be a weak form of conditional deontic explosion: given that something is somewhat desirable (passing a course with a grade of C, say) and given that the most desirable outcome (the grade of A) is unavailable, the somewhat desirable outcome becomes obligatory.

We then show that despite this paradox, the the systems of Carmo and Jones have interesting mathematical content. For the strongest system we provide a full classification of its models; for weaker versions we characterize the least models (under inclusion) satisfying the axioms and basic contrary-to-duty assumptions.

\section{The paradox}
	Let $W$ be a finite set of possible worlds of a given model, and let $\mathscr P$ denote the power set operation.

 %    \textsc{Carmo and Jones}\footnote{
	% 	Conditions 5(a)--(d) and 5(e) are first introduced on pages 331 and 341, respectively, of~\cite{CJ96}, where ob is called pi.
	% 	5(a)--(d) are also given in~\cite{CJ02}, page 291, with 5(e) on page 319.
	% 	The conditions 5(a)(b)(d)(e) and a condition (c$^*$) are given in~\cite{CJ13}, page 590.
	% } gave several
    
   Combining the systems from several papers, the full list of Carmo and Jones'
    conditions
	on a function $ob : \mathscr P(W) \to \mathscr P(\mathscr P(W))$ is as follows. 
	\begin{itemize}
		 \item[5(a)] Axiom 5(a): \fivea (Obligations cannot be impossible to fulfil.)
		\item[5(b)] Axiom 5(b): \fiveb
		(Whether $Z$ is obligatory in context $X$ depends only on $Z$ through $Z\cap X$.)
         \item[5(c)] Axiom 5(c$^-$) (2002, page 319; finite version of 5(c$^*$), 2013, which is called 5(c) in 2022): \fivecweak (Closure under intersection.)
         \item[5(c)] Axiom 5(c) (1997 and 2002, page 287): \fivec (Closure under intersection, even for disjoint sets.)
		 \item[5(d)] Axiom 5(d): \fived (If $Y$ is obligatory in context $X$ then a form of the material conditional ``$X \to Y$'' holds in a wider context.)
		\item[5(e)] Axiom 5(e): \fivee
        (If $Z$ is obligatory in the context $X$ then it remains obligatory in any subcontext in which it is still possible.)
	\item[5(f)] Axiom 5(f): \fivef
    (If $X$ is obligatory in both of the contexts $Y$ and $Z$, then it is obligatory in their union.)
	\item[5(g)] Axiom 5(g): \fiveg
    \end{itemize}

% In this file you should put the actual content of the blueprint.
% It will be used both by the web and the print version.
% It should *not* include the \begin{document}
%
% If you want to split the blueprint content into several files then
% the current file can be a simple sequence of \input. Otherwise It
% can start with a \section or \chapter for instance.

Carmo and Jones gave several axioms on their function $\mathrm{ob}$.


\begin{definition}\label{CJ5a}
	\lean{A5}
	\leanok
	Axiom 5(a) for a function $\ob: \Pow(U) \to \Pow(\Pow(U))$ says that $\emptyset\notin\ob(X)$ for all $X\in\Pow(U)$.
\end{definition}



\begin{definition}\label{CJ5b}
	\lean{B5}
	\leanok
	Axiom 5(b) for a function $\ob: \Pow(U) \to \Pow(\Pow(U))$ says that
	\[
	∀ X\, Y\, Z, Z ∩ X = Y ∩ X \to (Z ∈ \ob X ↔ Y ∈ \ob X)
	\]

\end{definition}

\begin{definition}\label{CJ5cstar}
	\lean{CJ5c_star}
	\leanok
	The weak, but potentially infinite, version of Axiom 5(c) for a function $\ob: \Pow(U) \to \Pow(\Pow(U))$ says that
	\[
    \forall X, \forall \beta \subseteq \ob X, \beta \ne \emptyset \rightarrow \bigcap \beta \cap X \ne \emptyset
	\rightarrow \bigcap \beta \in \ob X.	
	\]

\end{definition}

\begin{definition}\label{CJ5cstar}
	\lean{C5Strong}
	\leanok
	The strong version of Axiom 5(c) for a function $\ob: \Pow(U) \to \Pow(\Pow(U))$ says that
	\fivec
\end{definition}
 
 
\begin{definition}\label{CJ5d}
	\lean{D5}
	\leanok
	Axiom 5(d) for a function $\ob: \Pow(U) \to \Pow(\Pow(U))$ says that
	\[
	  ∀ (X, Y, Z\subseteq U), Y ⊆ X \to Y ∈ ob X \to X ⊆ Z \to Z \ X ∪ Y ∈ ob Z
	\]

\end{definition}

 
 
\begin{definition}\label{CJ5bd}
	\lean{CJ5bd}
	\leanok
	Axiom 5(bd) for a function $\ob: \Pow(U) \to \Pow(\Pow(U))$ says that
	$\forall (X Y Z \in\Pow U), Y \in \ob X \land X \subseteq Z \to Z \setminus X \cup Y \in \ob Z$.
\end{definition}

\begin{definition}\label{E5}
	\lean{E5}
	\leanok
	Axiom 5(e) for a function $\ob: \Pow(U) \to \Pow(\Pow(U))$ says that
\fivee
\end{definition}



\begin{theorem}\label{bd5}
	\lean{bd5}
	\leanok
	\uses{CJ5bd, CJ5b, CJ5d}
	If $\ob$ satisfies 5(b) and 5(d) then it satisfies 5(bd).
\end{theorem}

\begin{definition}\label{CJ5f}
	\lean{CJ5f}
	\leanok
	Axiom 5(f) says
	\[
		\forall \beta \subseteq \Pow(U), \beta \ne \emptyset \rightarrow
		\forall X, \text{ if} \forall Z \in \beta, X \in \ob Z \text{ then } X \in \ob \bigcup \beta.
	\]
\end{definition}
While 5(f) allows for infinite sets, the finite version can be stated more simply.

\begin{theorem}[Lemma II.2.2 of Carmo and Jones]\label{II22}
	\lean{II_2_2}
	\leanok
	\uses{CJ5a, CJ5b, CJ5cstar,CJ5d}
	If $\ob$ satisfies axioms 5(a,b,c*,d) then $\ob$ satisfies axiom 5(f).
\end{theorem}

\begin{definition}
	\lean{ob₂}
	\leanok
The 32 models $\ob_2(b)$, where $b$ is a Boolean vector of length 5, are defined by $B\in\ob_2(A)$ iff
the following are true:
\begin{itemize}
	\item $A \cap B \ne \emptyset$,
	\item $(B = \{0\} \to b 0)$,
	\item $(B = \{1\} \to b 1)$, and
	\item if $B = \{0,1\}$ then
		\begin{itemize}
			\item $(A = \{0\} \to b 3)$,
			\item $(A = \{1\} \to b 4)$, and
			\item $(A = \{0,1\} \to b 2)$.
		\end{itemize}
\end{itemize}
\end{definition}

Using this ob₂ we prove that
\begin{theorem}
	\lean{do_not_imply_5d_or_5f}
	\uses{ob₂}
	\leanok
There is a model where $5(a)(b)(c)(e)(g)$ all hold but $5(d)$ and $5(f)$
do not hold.
\end{theorem}

We also prove
\begin{theorem}
\lean{do_not_imply_5e}
\leanok
∃ ob, A5 ob ∧ B5 ob ∧ C5 ob ∧ D5 ob ∧ ¬ E5 ob ∧ F5 ob ∧ G5 ob.
\end{theorem}

\section{Canonical models of Carmo and Jones' systems}

We show that the two approaches sketched in [Kjos-Hanssen 2017] are both consistent with [Carmo Jones 2022]
\cite{MR4500520}.

Preferably, we let F(X) = X ∩ A for a fixed set A.

However, to incorporate contrary-to-duty obligations we introduce a predicate B,
where A worlds, A ⊆ B, are the best and B \setminus A worlds the second best.

Thus, if X ∩ A = ∅ but X ∩ B ≠ ∅, we let F(X) = X ∩ B.



\begin{table}[h]
    \centering
    \begin{tabular}{lcccc}
        \toprule
        Axiom \textbackslash\ Model & \texttt{canon} & \texttt{canon\_II} & \texttt{canon₂} & \texttt{canon₂\_II} \\
        \midrule
        A & ✓ & ✓ & ✓ & ✓ \\
        B & ✓ & ✓ & ✓ & ✓ \\
        C & ✓ & ✓ & ✓ & ✓ \\
        D & \textit{thus} ✓ & × & ✓ & \textit{thus} × \\
        E & × & ✓ & \textit{thus} × & ✓ \\
        F & ✓ & ✓ & ✓ & ×! \\
        G & ✓ & ✓ & ×! & ✓ \\
        \bottomrule
    \end{tabular}
    \caption{Results about which axioms hold in which model.}
    \label{tab:example}
\end{table}

\begin{definition}
	\lean{canon}
	\leanok
Given n : ℕ and A : Finset (Fin n)),
T $\in$ canon(A,S) iff S ∩ A $\ne$ ∅ and S ∩ A ⊆ T.
\end{definition}

\bibliographystyle{plain}
\bibliography{deontic}
