% In this file you should put the actual content of the blueprint.
% It will be used both by the web and the print version.
% It should *not* include the \begin{document}
%
% If you want to split the blueprint content into several files then
% the current file can be a simple sequence of \input. Otherwise It
% can start with a \section or \chapter for instance.
\begin{definition}\label{CJ5a}
	\lean{CJ5a}
	\leanok
	Axiom 5(a) for a function $\ob: \Pow(U) \to \Pow(\Pow(U))$ says that $\emptyset\notin\ob(X)$ for all $X\in\Pow(U)$.
\end{definition}



\begin{definition}\label{CJ5b}
	\lean{CJ5a}
	\leanok
	Axiom 5(b) for a function $\ob: \Pow(U) \to \Pow(\Pow(U))$ says that
	\[
	∀ (X Y Z : Set U), Z ∩ X = Y ∩ X → (Z ∈ ob X ↔ Y ∈ ob X)
	\]

\end{definition}
 
\begin{definition}\label{CJ5d}
	\lean{CJ5d}
	\leanok
	Axiom 5(d) for a function $\ob: \Pow(U) \to \Pow(\Pow(U))$ says that
	\[
	  ∀ (X Y Z : Set U), Y ⊆ X → Y ∈ ob X → X ⊆ Z → Z \ X ∪ Y ∈ ob Z
	\]

\end{definition}
 
 
\begin{definition}\label{CJ5bd}
	\lean{CJ5bd}
	\leanok
	Axiom 5(bd) for a function $\ob: \Pow(U) \to \Pow(\Pow(U))$ says that
	$\forall (X Y Z \in\Pow U), Y \in \ob X \land X \subseteq Z \to Z \setminus X \cup Y \in \ob Z$.
\end{definition}


\begin{theorem}\label{bd5}
	\lean{bd5}
	\leanok
	\uses{CJ5bd, CJ5b, CJ5d}
	If $\ob$ satisfies 5(b) and 5(d) then it satisfies 5(bd).
\end{theorem}
